\documentclass[final]{beamer}

% ====================
% Packages
% ====================
% Set custom size: 91.44cm x 121.92cm (approx 36x48 inches)
\usepackage[orientation=portrait,size=custom,width=91.44,height=121.92,scale=1.4,debug]{beamerposter}
\usepackage[utf8]{inputenc}
\usepackage[T1]{fontenc}
\usepackage[default]{lato}
\usepackage{tcolorbox} % For colored boxes
\usepackage{graphicx} % For images
\tcbuselibrary{skins}
% ====================
% Colors
% ====================
\definecolor{MilaMauve}{HTML}{662e7d}
\definecolor{MilaWhite}{HTML}{FFFFFF}
\definecolor{TextDark}{HTML}{333333}
\definecolor{LightGray}{HTML}{F0F0F0}

% ====================
% Styling
% ====================
\setbeamercolor{background canvas}{bg=MilaWhite}
\setbeamercolor{normal text}{fg=black}
\setbeamercolor{headline}{bg=MilaMauve,fg=MilaWhite}
\setbeamercolor{block title}{bg=LightGray,fg=black}
\setbeamercolor{block body}{bg=white,fg=black}

\renewcommand{\familydefault}{\sfdefault}
\renewcommand{\normalsize}{\fontsize{42}{50}\selectfont}
\normalsize

\newcommand{\PosterHeading}[1]{\fontsize{77}{84}\selectfont\bfseries #1}
\newcommand{\TakeawayFont}[1]{{\fontsize{78}{86}\selectfont\bfseries #1}}
\newcommand{\SectionHeading}[1]{\fontsize{48}{54}\selectfont\bfseries #1}

\tcbset{
    posterbox/.style={
        colback=white,
        boxrule=0pt,
        sharp corners,
        enhanced,
        left=0.6cm,
        right=0.6cm,
        top=0.4cm,
        bottom=0.6cm
    }
}

% Remove default navigation symbols
\setbeamertemplate{navigation symbols}{}

% Remove vertical spacing at the top of the document.
% put this before \begin{document}
\makeatletter
\addtobeamertemplate{frame begin}{%
  \vspace*{-\dimexpr\beamer@frametopskip\relax}%
}{}
\makeatother

% ====================
% Document
% ====================
\begin{document}

\begin{frame}[t]

    % --------------------------------------------------------
    % TOP HORIZONTAL BLOCK: TITLE (15% HEIGHT)
    % --------------------------------------------------------
 \setbeamercolor{headline}{bg=MilaMauve,fg=MilaWhite}

\begin{beamercolorbox}[wd=\paperwidth,ht=0.15\paperheight,dp=0ex,center,fg=milaWhite]{headline}
        \begin{minipage}[c][0.15\paperheight]{0.64\textwidth}
            \vspace*{-20cm} % same offset so both columns align
            {\raggedright\TakeawayFont{One sentence takeaway for people walking by your poster, but not stopping}}\\[0.8em]
            {\PosterHeading{Paper Title}}\\[0.4em]
            {\Large Author One \quad Author Two \quad Author Three}\\[0.2em]
            {\large Institution One \quad|\quad Institution Two}
        \end{minipage}%
        \hfill
        \begin{minipage}[c][0.15\paperheight]{0.3\textwidth}
            {\large Scan for the paper}\\[0.5em]
            \begin{center}
                \vspace*{-20cm} % same offset so both columns align
                \includegraphics[width=4.5cm]{example-image-1x1}\\[0.4em]
                {\small Generate a QR at \url{https://www.qrcode-monkey.com/}}\\
                {\small Set the foreground to white (\#FFFFFF) and the background to Mila purple (\#662e7d).}
            \end{center}
        \end{minipage}
    \end{beamercolorbox}

    \vspace{0.8cm}

    % --------------------------------------------------------
    % MAIN COLUMNS
    % --------------------------------------------------------
    \begin{columns}[t,totalwidth=\textwidth]

        % ========================================================
        % COLUMN 1: Summary (and Title info if needed)
        % ========================================================
        \begin{column}{0.30\paperwidth}

            % Summary guidance block
            \begin{tcolorbox}[posterbox]
                {\SectionHeading{SUMMARY}}

                \vspace{0.5em}

                This is for someone who will stop at your poster, but just wants to read. They should be able
                to understand the idea without talking to you, so keep it tight and approachable.

                \begin{itemize}
                    \item Lead with the abstract-level hook so curiosity kicks in.
                    \item Use the nice graphics around the text to reinforce your main message.
                    \item Bold a surprising metric or claim that makes people ask you a question.
                    \item Close with one sentence that suggests what they should do next (scan the QR, read the paper, etc.).
                \end{itemize}
            \end{tcolorbox}

        \end{column}

        % ========================================================
        % COLUMN 2: Paper Figures
        % ========================================================
        \begin{column}{0.30\paperwidth}

            % Central visual element
            \begin{tcolorbox}[posterbox]
                {\SectionHeading{FIGURES \& CALLOUTS}}

                \vspace{0.5em}

                \begin{center}
                    \includegraphics[width=0.92\linewidth, height=12cm, keepaspectratio]{example-image-a}

                    \vspace{0.4em}

                    {\large Figure Caption}\\
                    Explain the most important pipeline, diagram or comparison here. Keep it concise so you can elaborate while pointing at it.
                \end{center}

                \vspace{1cm}

                \begin{center}
                    \includegraphics[width=0.92\linewidth, height=12cm, keepaspectratio]{example-image-b}

                    \vspace{0.4em}

                    {\large Figure Caption}\\
                    Use a second figure for ablations, qualitative examples, or anything that benefits from a visual walkthrough.
                \end{center}

            \end{tcolorbox}

        \end{column}

        % ========================================================
        % COLUMN 3: Experiments, Details, Conclusions, Refs
        % ========================================================
        \begin{column}{0.30\paperwidth}

            \begin{tcolorbox}[posterbox]
                {\SectionHeading{EXPERIMENTS \& DETAILS}}

                \vspace{0.5em}

                This section is for people who stop at your poster and chat. Treat it like the extra figures
                or tables you would keep in a slide deck. When you stand on this side, you can point to the
                visuals while explaining the nuances.

                \vspace{0.8em}

                Highlight the experiment setup (datasets, key metrics, ablations) in short paragraphs instead
                of long sentences. Keep a punchy takeaway for each bullet so you can guide the conversation:
                \begin{itemize}
                    \item Talk through the evaluation protocol and why it matters.
                    \item Call out the delta between your method and the baseline.
                    \item Mention qualitative wins or failure cases that invite discussion.
                \end{itemize}

                \vspace{0.8em}

                If Summary feels too large and this column too small, shift the column widths until it balances
                for your specific content.
            \end{tcolorbox}

            \vspace{1cm}

            \begin{tcolorbox}[posterbox]
                {\SectionHeading{REFERENCES}}

                [1] Author et al. (2024) ``Paper Title''. Journal/Conf.\par
                [2] Author et al. (2023) ``Another Paper''. Journal/Conf.
            \end{tcolorbox}

            \vspace{1cm}
        \end{column}

    \end{columns}

    % --------------------------------------------------------
    % BOTTOM LOGO STRIP (FIXED TO BOTTOM MARGIN)
    % --------------------------------------------------------
    \vfill
    \begin{center}
        {\setlength{\tabcolsep}{0.75cm}%
        \begin{tabular}{ccc}
            \includegraphics[height=10cm]{../img/udem.pdf} &
            \includegraphics[height=10cm]{../img/mila_black.png} &
            \includegraphics[height=10cm]{../img/mcgill.png}
        \end{tabular}}
    \end{center}
    \vspace*{0.5cm}

\end{frame}
\end{document}
